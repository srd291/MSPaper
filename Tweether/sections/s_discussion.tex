\section{Discussion}

%We used the large data set to determine if there was any noticeable correlation, and we filter the data set to determine if this could provide a better indication of correlation. We saw that most of the tweets were positive in colder and warmer temperatures, however, the number of tweets increased as the temperature rose. We observed the number of positive tweets percentage wise was much higher when the temperature rose. Even though, our initial aim was to find a distinct answer to the question: Does the weather affect mood?, we were not able to gain concrete results. We were able to find other patterns from our analysis.
%
%We were able to determine the possible sentiment of users in select regions, even with lower a population. The results show ......
%
%Correlating tweets to weather patterns is not a trivial task and requires multiple steps which can create errors at each step. Having a multi-step process the chance of error from classification can occur.

There are a few aspects in regards to tweets that enforces a limit as to what we can assess. One limitation from tweets is that with a state population as small as Nebraska the chance of a user with their location enabled is relatively low. This limits the amount of tweets we can attain each hour. Tweet length imposes another limitation because the complete expression of a user's feelings can be curbed in some situations. Other than Twitter's limits we need to examine the suppressed feelings of users. As Twitter is a social media hub, the need to express oneself in 140 characters may place a mask to what the user may be feeling. As proposed before there is no distinct way to determine the true feelings of a person~\cite{hannak2012tweetin}. Given this phenomena, we experimented the usage of only weather related tweets because there is a lower chance of users masking their emotions. Even in this case we saw that the majority of users use their tweets in a positive manner, so to expect a negative tweet is cynical in some sense.

Classifying the sentiment always brings errors. Humans are not known to be the most accurate at identifying the sentiment of written statements~\cite{pangthumbs}. %When bringing in our own classifier, we believe that it can be improved. 
There are always certain words that can make the sentiment have a positive value instead of negative. Tweets are known to have sarcasm in terms of introducing a form of satire or irony~\cite{riloff2013sarcasm}. In situations like this there are some misclassifications of sentiment.

%We believe that using the current weather limits us on other seasons which are approaching. So far, 
We have experienced winter and spring weather in our current study. When we have the change of seasons it is easy to see that there is some correlation with SAD. Like most of the population, the end of winter brings new changes, and thus new feelings. We will study the sentiment changes when summer, fall, or winter approaches. Nebraska is known for its intense heat and harsh winters, but these seasons also entail vacations and family gatherings. Analyzing these aspects can also provide more insight regarding our data sets and other filtering processes that we may need.
