\section{Discussion}

We used a large data set to determine if there was any noticeable correlation, and we filter the data set to determine if this could provide a better indication of correlation. We saw that most of the tweets were positive in colder and warmer temperatures, however, the number of tweets increased as the temperature rose. We observed the number of positive tweets percentage wise was much higher when the temperature rose. Even though, our initial aim was to find a distinct answer to the question: Does the weather affect mood?, we were not able to gain concrete results. We were able to find other patterns from our analysis.

Our bundling methods created smooth curves to create an aesthetic design. ....

We were able to determine the possible sentiment of users in select regions, even with lower a population. The results show ......

Correlating tweets to weather patterns is not a trivial task and requires multiple steps which can create errors at each step. Having a multi-step process the chance of error from classification can occur.

There are a few items in regards to tweets that enforces a limit as to what we can assess. One drawback from tweets is that with a state population as small as Nebraska the chance of a user with their location enabled is minute. This drastically limits the amount of tweets we can attain each hour. Another limitation in some situations is tweet length since this may cause the complete expression of a user's feelings to be curbed. Other than Twitters limits we need to examine the suppressed feelings of users. Since Twitter is a social media hub, the need to express oneself in 140 characters may place a mask to what the user may be feeling. As proposed before there is no distinct way to determine the true feelings of a person.[3] Due to this phenomena we proposed the usage of only weather related tweets, due to there being a low chance of users masking their emotions. Even with this we saw that the majority of users use their tweets in a positive manner, so to expect a negative tweet is cynical in some sense.

Classifying the sentiment always brings error. Humans aren't known to be the most accurate at identifying the sentiment of written statements. \cite{pangthumbs} When bringing in our own classifier, we believe that it can be improved. There are always certain words that can make the sentiment have a positive value instead of negative. Tweets are known to have sarcasm in them to introduce a form of satire or irony. \cite{riloff2013sarcasm} In situations like this there are some misclassifications of sentiment.

We believe that using the current weather limits us on other seasons which are approaching. So far, we have experienced winter and spring weather. When we have the change of seasons it's easy to see that there is some correlation with SAD. Like most of the population, the end of winter brings new changes, and thus new feelings. However, we don't know how the sentiment will change when summer, fall, or winter approaches. Nebraska is known for its intense heat and harsh winters, but these seasons also entail vacations and family gatherings. Analyzing these things can also provide more insight regarding our data sets, and what other filtering processes we may need.
